\documentclass{article}
\usepackage{amsmath}
\usepackage{amssymb}
\usepackage[utf8]{inputenc}

\title{تحليل التشتت}

\begin{document}

\maketitle

بسم الله الرحمن الرحيم

\section{المقدمة}

في هذه المحاضرة، سنستعرض مفهوم تحليل التشتت مع التركيز على \textbf{التباين} و\textbf{الانحراف المعياري} من البيانات وجداول التكرار. سنتناول كيفية حساب هذه القيم خطوة بخطوة، مع أمثلة توضيحية لتعزيز الفهم.

\section{القسم الأول: التباين والانحراف المعياري من البيانات}

\subsection{1. تعريفات أساسية}

\subsubsection{1.1 التباين (Variance) - \textbf{الصيغة 1}}
$$
S^2 = \frac{∑(x - \bar{x})^2}{n}
$$

\subsubsection{1.2 الانحراف المعياري (Standard Deviation) - \textbf{الصيغة 2}}
$$
S = \sqrt{\frac{∑(x - \bar{x})^2}{n}}
$$

\subsection{2. مثال 1: حساب التباين والانحراف المعياري}

\textbf{البيانات:} 1، 4، 0، 3، 4، 6

\subsubsection{2.1 حساب الوسط الحسابي ($\bar{x}$)}
$$
\bar{x} = \frac{1 + 4 + 0 + 3 + 4 + 6}{6} = 3
$$

\subsubsection{2.2 حساب الانحرافات ($D$)}
$$
D = (1 - 3), (4 - 3), (0 - 3), (3 - 3), (4 - 3), (6 - 3) = -2, 1, -3, 0, 1, 3
$$

\subsubsection{2.3 حساب مربعات الانحرافات ($D^2$) وجمعها}
$$
∑D^2 = 4 + 1 + 9 + 0 + 1 + 9 = 24
$$

\subsubsection{2.4 حساب التباين ($S^2$)}
$$
S^2 = \frac{24}{6} = 4
$$

\subsubsection{2.5 حساب الانحراف المعياري ($S$)}
$$
S = \sqrt{4} = 2
$$

\textbf{النتيجة:}

\begin{itemize}
\item \textbf{التباين} = 4
\item \textbf{الانحراف المعياري} = 2
\end{itemize}

\section{القسم الثاني: التباين والانحراف المعياري من جدول التكرار}

\subsection{1. تعريفات أساسية}

\subsubsection{1.1 حساب الوسط الحسابي ($\bar{x}$) من جدول التكرار}
$$
\bar{x} = \frac{∑m.f}{∑f}
$$

\subsubsection{1.2 حساب التباين ($S^2$) من جدول التكرار - \textbf{الصيغة 3}}
$$
S^2 = \frac{∑f.(m - \bar{x})^2}{∑f}
$$

\subsubsection{1.3 حساب الانحراف المعياري ($S$)}
$$
S = \sqrt{S^2}
$$

\subsection{2. مثال 2: حساب التباين والانحراف المعياري من جدول التكرار}

\textbf{الجدول:}

\begin{tabular}{|c|c|c|c|c|}
\hline
الفئة (C) & 1 - 5 & 5 - 9 & 9 - 13 & 13 - 17 \
\hline
\textbf{التكرار (f)} & 2 & 4 & 1 & 3 \
\hline
\end{tabular}

\subsection{3. الخطوات الحسابية}

\subsubsection{3.1 حساب النقاط الوسطية (m)}

\begin{itemize}
\item \textbf{الفئة 1-5:}
$$
m_1 = \frac{1 + 5}{2} = 3
$$

\item \textbf{الفئة 5-9:}
$$
m_2 = \frac{5 + 9}{2} = 7
$$

\item \textbf{الفئة 9-13:}
$$
m_3 = \frac{9 + 13}{2} = 11
$$

\item \textbf{الفئة 13-17:}
$$
m_4 = \frac{13 + 17}{2} = 15
$$
\end{itemize}

\subsubsection{3.2 ملء الجدول بالحسابات}

\begin{tabular}{|c|c|c|c|c|c|c|}
\hline
الفئة (C) & f & m & m.f & (m - $\bar{x}$) & $(m - \bar{x})^2$ & f·$(m - \bar{x})^2$ \
\hline
1 - 5 & 2 & 3 & 6 & -6 & 36 & 72 \
\hline
5 - 9 & 4 & 7 & 28 & -2 & 4 & 16 \
\hline
9 - 13 & 1 & 11 & 11 & 2 & 4 & 4 \
\hline
13 - 17 & 3 & 15 & 45 & 6 & 36 & 108 \
\hline
\textbf{المجموع} & 10 & & 90 & 0 & & 200 \
\hline
\end{tabular}

\subsubsection{3.3 حساب الوسط الحسابي ($\bar{x}$)}
$$
\bar{x} = \frac{90}{10} = 9
$$

\subsubsection{3.4 حساب التباين ($S^2$)}
$$
S^2 = \frac{200}{10} = 20
$$

\subsubsection{3.5 حساب الانحراف المعياري ($S$)}
$$
S = \sqrt{20} ≈ 4.47
$$

\textbf{النتيجة:}

\begin{itemize}
\item \textbf{التباين} = 20
\item \textbf{الانحراف المعياري} ≈ 4.47
\end{itemize}

\section{الخاتمة}

في هذه المحاضرة، تعلمنا كيفية حساب \textbf{التباين} و\textbf{الانحراف المعياري} من البيانات الخام وجداول التكرار. هذه القيم تساعدنا في فهم مدى تشتت البيانات حول الوسط الحسابي، مما يوفر رؤية أعمق حول توزيع البيانات وتحليلها إحصائيًا.

شكراً جزيلاً على متابعتكم!

\end{document}
